\usepackage[top=2cm, bottom=2cm, right=1cm, left=1cm]{geometry}
\usepackage[french]{babel}
\usepackage[T1]{fontenc}
\usepackage[utf8]{inputenc}
\usepackage{tikz,pgf}
\usepackage{multicol} %ecrire sur plusieurs colonnes
\frenchbsetup{StandardLists=true} %pour avoir des points et non des traits dans les listes
\usepackage{mathrsfs,amssymb,amsmath}
\usepackage{enumerate}
\usepackage[bf]{caption}
\usepackage{stmaryrd} %crochet intervalles entiers
\usepackage{ifthen}



%%%%%%%%%%%%%%%%%%%%%%%%%%%%
%%%%%%%%%%%%%%%%%%%%%%%%%%%%
%%%%%%%  Questions %%%%%%%%%
%%%%%%%%%%%%%%%%%%%%%%%%%%%%
%%%%%%%%%%%%%%%%%%%%%%%%%%%%
\newcounter{NumQLigne}
\setcounter{NumQLigne}{1}
\newcounter{NumQColonne}
\setcounter{NumQColonne}{1}
\newcommand{\question}{
\bigskip  \noindent \textbf{Question \Alph{NumQLigne}\theNumQColonne}.
\ifnum\theNumQColonne<\n
\stepcounter{NumQColonne} 
\else 
\stepcounter{NumQLigne}
\setcounter{NumQColonne}{1}
\fi
}

\newcounter{NumRLigne}
\setcounter{NumRLigne}{1}
\newcounter{NumRColonne}
\setcounter{NumRColonne}{1}
\newcommand{\reponse}{
\bigskip  \noindent \textbf{Réponse \Alph{NumRLigne}\theNumRColonne}.
\ifnum\theNumRColonne<\n
\stepcounter{NumRColonne} 
\else 
\stepcounter{NumRLigne}
\setcounter{NumRColonne}{1}
\fi
}



%%%%%%%%%%%%%%%%%%%%%%%%%%%%
%%%%%%%%%%%%%%%%%%%%%%%%%%%%
%%%%%%%  Raccourcis %%%%%%%%
%%%%%%%%%%%%%%%%%%%%%%%%%%%%
%%%%%%%%%%%%%%%%%%%%%%%%%%%%

\newcommand{\C}{\mathbb{C}}
\newcommand{\Q}{\mathbb{Q}}
\newcommand{\Z}{\mathbb{Z}}
\newcommand{\N}{\mathbb{N}}
\newcommand{\K}{\mathbb{K}}
\newcommand{\R}{\mathbb{R}}
\newcommand{\Hi}{\mathbb{H}}
\renewcommand{\P}{\mathbb{P}}
\renewcommand{\phi}{\varphi}
\newcommand{\esp}{\mathbb{E}}
\newcommand{\ds}{\displaystyle}
\newcommand{\partie}{\mathcal{P}}
\newcommand{\mat}[1]{\mathcal{M}_{#1}(\R)}
\newcommand{\sym}[1]{\mathcal{S}_{#1}(\R)}
\newcommand{\tra}{{}^t \!}
\renewcommand{\epsilon}{\varepsilon}
\newcommand{\iii}[4]{\int_{#1}^{#2} #3 \, \mathrm{d} {#4}}
