\documentclass{article}
\newcommand\n{5} %Côté grille réponse
\usepackage[top=2cm, bottom=2cm, right=1cm, left=1cm]{geometry}
\usepackage[french]{babel}
\usepackage[T1]{fontenc}
\usepackage[utf8]{inputenc}
\usepackage{tikz,pgf}
\usepackage{multicol} %ecrire sur plusieurs colonnes
\frenchbsetup{StandardLists=true} %pour avoir des points et non des traits dans les listes
\usepackage{mathrsfs,amssymb,amsmath}
\usepackage{enumerate}
\usepackage[bf]{caption}
\usepackage{stmaryrd} %crochet intervalles entiers
\usepackage{ifthen}



%%%%%%%%%%%%%%%%%%%%%%%%%%%%
%%%%%%%%%%%%%%%%%%%%%%%%%%%%
%%%%%%%  Questions %%%%%%%%%
%%%%%%%%%%%%%%%%%%%%%%%%%%%%
%%%%%%%%%%%%%%%%%%%%%%%%%%%%
\newcounter{NumQLigne}
\setcounter{NumQLigne}{1}
\newcounter{NumQColonne}
\setcounter{NumQColonne}{1}
\newcommand{\question}{
\bigskip  \noindent \textbf{Question \Alph{NumQLigne}\theNumQColonne}.
\ifnum\theNumQColonne<\n
\stepcounter{NumQColonne} 
\else 
\stepcounter{NumQLigne}
\setcounter{NumQColonne}{1}
\fi
}

\newcounter{NumRLigne}
\setcounter{NumRLigne}{1}
\newcounter{NumRColonne}
\setcounter{NumRColonne}{1}
\newcommand{\reponse}{
\bigskip  \noindent \textbf{Réponse \Alph{NumRLigne}\theNumRColonne}.
\ifnum\theNumRColonne<\n
\stepcounter{NumRColonne} 
\else 
\stepcounter{NumRLigne}
\setcounter{NumRColonne}{1}
\fi
}



%%%%%%%%%%%%%%%%%%%%%%%%%%%%
%%%%%%%%%%%%%%%%%%%%%%%%%%%%
%%%%%%%  Raccourcis %%%%%%%%
%%%%%%%%%%%%%%%%%%%%%%%%%%%%
%%%%%%%%%%%%%%%%%%%%%%%%%%%%

\newcommand{\C}{\mathbb{C}}
\newcommand{\Q}{\mathbb{Q}}
\newcommand{\Z}{\mathbb{Z}}
\newcommand{\N}{\mathbb{N}}
\newcommand{\K}{\mathbb{K}}
\newcommand{\R}{\mathbb{R}}
\newcommand{\Hi}{\mathbb{H}}
\renewcommand{\P}{\mathbb{P}}
\renewcommand{\phi}{\varphi}
\newcommand{\esp}{\mathbb{E}}
\newcommand{\ds}{\displaystyle}
\newcommand{\partie}{\mathcal{P}}
\newcommand{\mat}[1]{\mathcal{M}_{#1}(\R)}
\newcommand{\sym}[1]{\mathcal{S}_{#1}(\R)}
\newcommand{\tra}{{}^t \!}
\renewcommand{\epsilon}{\varepsilon}
\newcommand{\iii}[4]{\int_{#1}^{#2} #3 \, \mathrm{d} {#4}}



\begin{document}
\question Un polynôme $P \in \R[x]$ de degré 3, de coefficient dominant 2 et possédant 3 racines distinctes $x_1$, $x_2$ et $x_3$, vérifie : 
$$
\forall x \in \R, \quad P(x) = 2x^3 - ( x_1 + x_2 + x_3) x^2 + (x_1x_2 + x_1x_3 + x_2x_3) x - x_1 x_2 x_3 .
$$

\reponse Faux

\question Un polynôme $P \in \R[x]$ sans racine n'est pas factorisable.

\reponse Faux

\question $\exists x \in \R, \quad x^4 + x^3 + x^2 - 1 = x^3 + 3x^2 - 2$.

\reponse Vrai

\question $\forall x \in \R, \quad x^4 + x^3 + x^2 - 1 = x^3 + 3x^2 - 2$.

\reponse Faux

\question On pose pour tout $ n \in \N, S_n = \ds\sum_{k=0}^n \dfrac{1}{k+1}.$ Soit $n \in \N$, la somme $S_n$ contient $n+1$ termes.

\reponse Vrai

\question On pose pour tout $ n \in \N, S_n = \ds \sum_{k=0}^n \dfrac{1}{k+1},$ on a $S_{2n} = \ds\sum_{k=0}^{n} \dfrac{1}{2k+1}.$

\reponse Faux

\question Soit $b \in \R$. On considère le système linéaire suivant d'inconnues $x$, $y$ et $z$ :

$ \left\lbrace \begin{array}{ccccccc}
x & + &  y & + &  z &  = & 0  \\
x & + & 2y & + & 3z &  = & 0  \\
x & + & 3y & + & bz &  = & 0  
\end{array} \right. $

Il existe une valeur de $b$ pour laquelle le système n'a pas de solution.

\reponse Faux

\question Soit $b \in \R$. On considère le système linéaire suivant d'inconnues $x$, $y$ et $z$ :

$ \left\lbrace \begin{array}{ccccccc}
x & + &  y & + &  z &  = & 0  \\
x & + & 2y & + & 3z &  = & 0  \\
x & + & 3y & + & bz &  = & 0  
\end{array} \right. $

Il existe une valeur de $b$ pour laquelle le système a exactement une solution.

\reponse Vrai

\question Soit $b \in \R$. On considère le système linéaire suivant d'inconnues $x$, $y$ et $z$ :

$ \left\lbrace \begin{array}{ccccccc}
x & + &  y & + &  z &  = & 0  \\
x & + & 2y & + & 3z &  = & 0  \\
x & + & 3y & + & bz &  = & 0  
\end{array} \right. $

Il existe une valeur de $b$ pour laquelle le système a strictement plus d'une solution.

\reponse Vrai

\question Lors de l'oral d'un examen, une urne contient 4 questions indiscernables au toucher. Quatre étudiants piochent successivement et sans remise une question. Antoine, l'un des étudiants, à fait l'impasse sur l'une des questions.

On note $r$ le rang de passage d'Antoine et $p$ la probabilité qu'il pioche son impasse.

Le nombre $p$ ne dépend pas du rang de passage $r$.


\reponse Vrai

\question Lors de l'oral d'un examen, une urne contient 4 questions indiscernables au toucher. Quatre étudiants piochent successivement et sans remise une question. Antoine, l'un des étudiants, à fait l'impasse sur l'une des questions.

On note $r$ le rang de passage d'Antoine et $p$ la probabilité qu'il pioche son impasse.

Plus $r$ est grand plus $p$ est petit (strictement).

\reponse Faux

\question Lors de l'oral d'un examen, une urne contient 4 questions indiscernables au toucher. Quatre étudiants piochent successivement et sans remise une question. Antoine, l'un des étudiants, à fait l'impasse sur l'une des questions.

On note $r$ le rang de passage d'Antoine et $p$ la probabilité qu'il pioche son impasse.

Si $r=1$ alors $p = \dfrac{1}{4}$.

\reponse Vrai

\question Lors de l'oral d'un examen, une urne contient 4 questions indiscernables au toucher. Quatre étudiants piochent successivement et sans remise une question. Antoine, l'un des étudiants, à fait l'impasse sur l'une des questions.

On note $r$ le rang de passage d'Antoine et $p$ la probabilité qu'il pioche son impasse.


Si $r=3$ alors $p = \dfrac{1}{2}$.

\reponse Faux

\question Soit $f : \R \to \R$ une fonction. $f$ est dite surjective si chaque image admet au moins un antécédent.  

\reponse Faux

\question La fonction $g : \left\{ \begin{array}{rcl}
\R  & \to & \R^+_* \\
x   & \mapsto & \exp\left( 2x-1 \right)
\end{array} \right.$ est bijective et sa fonction réciproque est 

$$ \left\{ \begin{array}{rcl}
\R^*_+  & \to & \R \\
x   & \mapsto &  \frac{\ln(x)}{2}  +1 
\end{array} \right.
$$


\reponse Faux

\question  On a 

$$
\left(\begin{array}{rrr}
1 & 0 & -1 \\
 0  & 2 & 4 \\
 0  & 0 & 3 \\
 \end{array} \right)
 \left(\begin{array}{rrr}
 -1 & 4 & 2 \\
 0  & -1 & 0 \\
 0  & 0 & 2 \\
 \end{array} \right) =
 \left(\begin{array}{rrr}
 -1 & 4 & 0 \\
 0  & -2 & 8 \\
 0  & 0 & 6 \\
 \end{array} \right).
$$

\reponse Vrai

\question Soient $n \in \N^*$, $A$ et $B$ deux matrices de $\mat{n}$.

$$
(A+B)^2 - (A-B)^2 = 4AB
$$

\reponse Faux

\question Soit $A \in \mat{3}$ telle que $A  \begin{pmatrix} 0 \\ 1 \\ 2 \end{pmatrix} = \begin{pmatrix} 1 \\ 0 \\ 0 \end{pmatrix}$ et  $A  \begin{pmatrix} 3 \\ 4 \\ 5 \end{pmatrix} = \begin{pmatrix} 0 \\ 1 \\ 0 \end{pmatrix}$, alors on peut être certain que 

$$
A  \begin{pmatrix} 6 \\ 7 \\ 8 \end{pmatrix} = \left(\begin{array}{r} -1 \\ 2 \\ 0 \end{array} \right).
$$

\reponse Vrai

\question On possède un jeu de 52 cartes. Le nombre de mains de 5 cartes avec au moins un valet est $\ds\binom{4}{1} \binom{51}{4}.$

\reponse Faux

\question $\ds \lim_{x \to +\infty} \dfrac{e^{2 \ln(x)}}{x^3} = 0$.

\reponse Vrai

\question La fonction $f : [-2, 1[ \cup ]1,3] \to \R$, dont la représentation graphique est donnée ci-dessous semble continue. 


\begin{center}
\definecolor{qqqqff}{rgb}{0.,0.,1.}
\definecolor{cqcqcq}{rgb}{0.7529411764705882,0.7529411764705882,0.7529411764705882}
\begin{tikzpicture}[line cap=round,line join=round,x=1.0cm,y=1.0cm]
\draw [color=cqcqcq,, xstep=1.0cm,ystep=1.0cm] (-2.412574139030266,-1.822748715648821) grid (3.6958420047469236,2.464524726878044);
\draw[->,color=black] (-2.412574139030266,0.) -- (3.6958420047469236,0.);
\foreach \x in {-2.,-1.,1.,2.,3.}
\draw[shift={(\x,0)},color=black] (0pt,2pt) -- (0pt,-2pt) node[below] {\footnotesize $\x$};
\draw[->,color=black] (0.,-1.822748715648821) -- (0.,2.464524726878044);
\foreach \y in {-1.,1.,2.}
\draw[shift={(0,\y)},color=black] (2pt,0pt) -- (-2pt,0pt) node[left] {\footnotesize $\y$};
\draw[color=black] (0pt,-10pt) node[right] {\footnotesize $0$};
\clip(-2.412574139030266,-1.822748715648821) rectangle (3.6958420047469236,2.464524726878044);
\begin{scriptsize}
\draw[color=qqqqff] (-1.7486158625327453,-0.75) node {$\mathcal{C}_f$};
\draw [fill=qqqqff] (-2.,-0.9092974268256817) circle (2.0pt);
\draw [fill=qqqqff] (3.,0.5) circle (2.0pt);
\draw[line width=1.5pt,color=qqqqff, smooth,samples=100,domain=-2:1] plot(\x,{sin(((\x)*180)/3.14)});
\draw[-,color=qqqqff,line width=1.5pt] (1,0.84+0.2) -- (1,0.84-0.2);
\draw[-,color=qqqqff,line width=1.5pt] (1.1,0.84+0.2) -- (1,0.84+0.2);
\draw[-,color=qqqqff,line width=1.5pt] (1.1,0.84-0.2) -- (1,0.84-0.2);
\draw[line width=1.5pt,color=qqqqff, smooth,samples=100,domain=1:3] plot(\x,{3/(\x*2)});
\draw[-,color=qqqqff,line width=1.5pt] (1,1.5+0.2) -- (1,1.5-0.2);
\draw[-,color=qqqqff,line width=1.5pt] (0.9,1.5+0.2) -- (1,1.5+0.2);
\draw[-,color=qqqqff,line width=1.5pt] (0.9,1.5-0.2) -- (1,1.5-0.2);
\end{scriptsize}
\end{tikzpicture}
\end{center}

\reponse Vrai

\question On dispose de trois urnes $U_1$, $U_2$ et $U_3$ opaques contenant des boules indiscernables au toucher. 

\begin{itemize}
\item $U_1$ en contient 2 rouges, 3 blanches et 5 vertes ;
\item $U_2$ en contient 2 rouges, 5 blanches et aucune verte ;
\item $U_3$ en contient 0 rouge, 3 blanches et 6 vertes.
\end{itemize}

On tire une boule dans $U_1$ que l'on place dans $U_2$, puis une dans $U_2$ que l'on place dans $U_3$ et enfin une dans $U_3$.

On veut calculer la probabilité que toutes les boules piochées soient verte.

On pense à priori à utiliser la formule des probabilités composées. 

\reponse Vrai

\question Soit $f : \R \to \R$ dont la courbe représentative  sur $[-4,4]$  est : 

\definecolor{ffqqqq}{rgb}{1.,0.,0.}
\definecolor{cqcqcq}{rgb}{0.7529411764705882,0.7529411764705882,0.7529411764705882}
\begin{center}
\begin{tikzpicture}[line cap=round,line join=round,x=1.0cm,y=1.0cm]
\draw [color=cqcqcq,, xstep=1.0cm,ystep=1.0cm] (-4.5,-2.75) grid (4.5,2.75);
\draw[->,color=black] (-4.5,0.) -- (4.5,0.);
\foreach \x in {-4.,-3.,-2.,-1.,1.,2.,3.,4.}
\draw[shift={(\x,0)},color=black] (0pt,2pt) -- (0pt,-2pt) node[below] {\footnotesize $\x$};
\draw[->,color=black] (0.,-2.75) -- (0.,2.75);
\foreach \y in {-2.,-1.,1.,2.}
\draw[shift={(0,\y)},color=black] (2pt,0pt) -- (-2pt,0pt) node[left] {\footnotesize $\y$};
\draw[color=black] (0pt,-10pt) node[right] {\footnotesize $0$};
\clip(-4.5,-2.75) rectangle (4.5,2.75);
\draw[line width=1.2pt,color=ffqqqq,smooth,samples=100,domain=0:4.5] plot(\x,{(\x)^(0.6)});
\draw[line width=1.2pt,color=ffqqqq,smooth,samples=100,domain=-4.5:0] plot(\x,{0-(-(\x))^(0.6)});
\begin{scriptsize}
\draw [fill=black] (-3.174001421660925,4.950488068249199) circle (2.5pt);
\draw[color=black] (-3.0489275193580196,5.244411738661027) node {$a = 0.6$};
\draw[color=ffqqqq] (1,2) node {\huge $\mathcal{C}_f$};
\end{scriptsize}
\end{tikzpicture}
\end{center}

On considère la suite $u : \left\{ \begin{array}{l}
u_0 = \frac{1}{2},\\
\forall n \in \N, u_{n+1} = f(u_n).\\
\end{array} \right.$ 

La suite $u$ semble converger.

\reponse Vrai

\question Soit $f$ une fonction continue définie sur $[0,1]$ telle qu'il existe $c \in [0,1]$ vérifiant $f(c) = 0.$

On peut affirmer que $f(0) f(1) \leq 0$.

\reponse Faux

\question Soit $f : [-4,4] \to \R$ une fonction bijective dont la courbe représentative est : 

\definecolor{ffqqqq}{rgb}{1.,0.,0.}
\definecolor{cqcqcq}{rgb}{0.7529411764705882,0.7529411764705882,0.7529411764705882}
\begin{center}
\begin{tikzpicture}[line cap=round,line join=round,x=1.0cm,y=1.0cm]
\draw [color=cqcqcq,, xstep=1.0cm,ystep=1.0cm] (-4.5,-2.75) grid (4.5,2.75);
\draw[->,color=black] (-4.5,0.) -- (4.5,0.);
\foreach \x in {-4.,-3.,-2.,-1.,1.,2.,3.,4.}
\draw[shift={(\x,0)},color=black] (0pt,2pt) -- (0pt,-2pt) node[below] {\footnotesize $\x$};
\draw[->,color=black] (0.,-2.75) -- (0.,2.75);
\foreach \y in {-2.,-1.,1.,2.}
\draw[shift={(0,\y)},color=black] (2pt,0pt) -- (-2pt,0pt) node[left] {\footnotesize $\y$};
\draw[color=black] (0pt,-10pt) node[right] {\footnotesize $0$};
\clip(-4.5,-2.75) rectangle (4.5,2.75);
\draw[line width=1.2pt,color=ffqqqq,smooth,samples=100,domain=-4:4] plot(\x,{1+0.05*(\x)^(3)});
\begin{scriptsize}
\draw [fill=black] (-3.174001421660925,4.950488068249199) circle (2.5pt);
\draw[color=black] (-3.0489275193580196,5.244411738661027) node {$a = 0.6$};
\draw[color=ffqqqq] (1,2) node {\huge $\mathcal{C}_f$};
\end{scriptsize}
\end{tikzpicture}
\end{center}

$f^{-1}$ semble dérivable.

\reponse Faux
\end{document}